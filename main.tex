\documentclass[letterpaper]{mc2025}

%%% Packages Required by Class (already included)
% fancyhdr
% lastpage
% titling
% titlesec
% ragged2e
% enumitem
% amsmath
% graphicx
% geometry
% newtxtext
% newtxmath
% hyperref
% cleveref
% caption
% authblk
% apptools
% appendix
% ifpdf
% epstopdf

%%% Some other useful packages
% \usepackage{tikz}
% \usepackage{color}
% \usepackage{subcaption}
% \usepackage{algcompatible}
% \usepackage{bm}
% \usepackage{array}

\usepackage[acronym]{glossaries}
%\makeglossaries
\include{acros}

\usepackage{xspace}
%\usepackage{graphicx}
\graphicspath{{images/}}

\usepackage{placeins}
\usepackage{booktabs} % nice rules (thick lines) for tables
\usepackage{array}
\usepackage{microtype} % improves typography for PDF

%\usepackage[hidelinks]{hyperref}
%\usepackage{caption}
\usepackage{subcaption}
\usepackage{hhline}
%\usepackage{amsmath}
%\usepackage{amssymb}
\usepackage{mathtools}
\allowdisplaybreaks
\usepackage{color}
\usepackage{multirow}
\usepackage{siunitx}
\usepackage{xfrac}
\usepackage{bm}
\usepackage{stmaryrd}

\usepackage{threeparttable, tablefootnote}

\usepackage{environ}
\makeatletter

\usepackage{tabularx}
\usepackage{float}
\usepackage{enumitem}
\setlist[itemize]{noitemsep}
\usepackage{diagbox}
\usepackage{courier}
\usepackage{pdflscape}

%\usepackage{cleveref}
\usepackage{datatool}
% \usepackage[numbers]{natbib}
\usepackage{notoccite}

\usepackage{nomencl} % If needed
\makenomenclature

\usepackage{tikz}
\usepackage{tkz-euclide}
\usetikzlibrary{positioning, arrows, decorations, shapes}
\usetikzlibrary{shapes.geometric,arrows}
\definecolor{illiniblue}{HTML}{B1C6E2}
\definecolor{illiniorange}{HTML}{f8c2a2}
\definecolor{green}{HTML}{c2e2b1}
\tikzstyle{process} = [rectangle, rounded corners, thick, minimum width=7cm, minimum height=1cm, text centered, draw=black, fill=illiniblue, text width=18em]
\tikzstyle{object} = [ellipse, minimum width=2cm, thick, minimum height=2.2cm, text centered, draw=black, fill=illiniorange, text width=12em]
\tikzstyle{decision} = [diamond, thick, aspect=2, minimum width=2cm, minimum height=2cm, text centered, draw=black, fill=green, text width=6em]
\tikzstyle{arrow} = [thick,->,>=stealth]

\title{A Hybrid $S_N$-Diffusion Method for Control Rod Modeling in Molten Salt Reactors}

%%% Authors (use arabic numbers: 1, 2, 3, etc. for affiliationNumber)
%%% \addAuthor{GivenName MiddleInitial. FamilyName}{affiliationNumber}
\addAuthor[smpark3@illinois.edu]{Sun Myung Park}{1}
\addAuthor{Kathryn D. Huff}{1}
\addAuthor{Madicken Munk}{2}
% To move the * for the corresponding author move the Email address for primary/corresponding author only. Move the * next to the appropriate name. Do not use all capital letters for any part of the author's name] and insert it before the Author addition ex. if corresponding author is A then adjust \addAuthor[email]{Author A}{1}

%%% Affiliations (from authblk)
%%% \addAffiliation{affiliationNumber}{Name of Institute, City, State/Country}
\addAffiliation{1}{University of Illinois Urbana-Champaign, Urbana, IL}
\addAffiliation{2}{Oregon State University, Corvallis, OR}


%%% Write text for abstract
%%% Most text modifying commands will work in abstract
\Abstract{%
This paper presents a hybrid $S_N$-diffusion method for accurate modeling of control rods in
molten salt reactors. Neutron diffusion methods struggle with strong neutron
absorptivity of control rods, while neutron transport methods are computationally prohibitive
for time-dependent reactor analyses. The proposed hybrid method combines the strengths of both
approaches by applying transport corrections generated using the $S_N$ method near control rods.
Six 1-D test cases with increasing complexity are used to verify the implementation and assess
performance. Results show that the hybrid method provides improved rod worth estimates compared to
pure diffusion models. The hybrid method iterative scheme demonstrates robustness in convergence
and converges rapidly within two outer iterations. Additionally, the method accurately reproduces
much of the reference transport correction distribution from a full $S_N$ calculation despite the
$P_1$ approximation for $S_N$-diffusion boundary coupling. Overall, the hybrid $S_N$-diffusion
method offers a promising avenue for accurate control rod modeling in time-dependent reactor
analyses. Further work is ongoing to extend the hybrid method to higher dimensional models and
time-dependent analyses.
}

%%% List up to 5 keywords separated by a comma
\keywords{control rod modeling, molten salt reactor, neutron transport, reactor physics}

%%% Provide a short title for the header on odd pages
\shortTitle{A Hybrid $S_N$-Diffusion Method for Control Rod Modeling in Molten Salt Reactors}

%%% Provide a short author listing for the header on even pages
\authorHead{S. Park, et al.}

%%% If LaTeX reports the line number of an error at \begin{document} it
%%%   is most likely due to an error in one of the commands above
\begin{document}

\section{Introduction}\label{sec:1}

\glspl*{MSR} are advanced reactors noted for strong passive safety due to their liquid fuel form,
negative temperature reactivity feedback, and low operating pressure. While some \gls*{MSR} concepts
completely eliminated control rods from their designs, many designs still retain control rods for
reactivity control and emergency scram. Accurate control rod modeling capability is essential for
studying reactor behavior in transient scenarios involving control rod movement.

Routine time-dependent and spatially-resolved reactor analyses remain in the domain of neutron
diffusion methods due to the large computational expense of neutron transport methods. Several
transport correction techniques exist in the literature for improving neutron flux and
$k_\text{eff}$ estimates of neutron diffusion methods augmented by correction parameters in
various forms
\cite{ronen_accurate_2004, pounders_diffusion_2009, kavenoky_sph_1978, fen_modelling_1992,
tamang_multilevel_2014, koebke_new_1980}.

In this paper, we propose a hybrid $S_N$-diffusion neutronics method to improve control rod
modeling in neutron diffusion solvers. In essence, the hybrid method is an iterative method that
adaptively applies the $S_N$ discrete ordinates neutron transport method on subregions containing
the control rod to obtain pointwise transport corrections for the diffusion method.
This paper presents the theoretical background and computational algorithm for the hybrid method
and verification results through several \gls*{MSR}-based 1-D test cases.

% Section \ref{sec:summary-nts-mtds} highlights the poor performance of neutron diffusion
% methods for calculating neutron fluxes near control rods. Strong neutron absorption in the control
% rod region produces a highly anisotropic neutron flux extending some distance outside the control
% rod. Neutron transport methods, which retain angular dependence of the neutron flux to various
% extents, generally fare better than neutron diffusion methods, which have isotropic diffusion
% coefficients. However, neutron transport methods are also generally more computationally expensive,
% given the increased dimensionality of the problem from the angular component. Adding an angular
% dimension to the existing geometric and neutron energy group dimensions dramatically
% increases the problem size and the computational resources necessary to solve the system. Many past
% efforts have tried introducing
% transport correction techniques to improve neutron flux and multiplication factor estimates in
% diffusion-based methods. Other than control rod regions, these techniques may also correct
% homogenization errors introduced by spatial homogenization of fuel assemblies and other
% structures within a reactor core. They invariably rely on neutron transport methods to generate
% transport corrections in the form of corrected diffusion coefficients
% \cite{bretscher_computing_1997, scherer_determination_1976, ronen_accurate_2004,
% pounders_diffusion_2009, kavenoky_sph_1978}, boundary conditions \cite{davison_influence_1951,
% pellaud_extrapolation_1968, fen_modelling_1992}, Eddington factors, or discontinuity factors
% \cite{koebke_new_1980}.

\section{Theory} \label{sec:theory}

% Section \ref{sec:saaf} provides the mathematical derivation for the discretized weak form of the
% \gls*{SAAF} $S_N$ neutron transport equations. Section \ref{sec:transport-correction} provides the
% derivations for the transport correction formulations we investigated for use in the hybrid method.
% Section \ref{sec:hybrid-algorithm} details the iteration algorithm for the hybrid method. Section
% \ref{sec:sn-bc} provides the boundary conditions formulated for the $S_N$ calculations in the
% hybrid method. In Section \ref{sec:buffer-region}, we discuss how transport corrections are handled
% in the hybrid method. Lastly, Section \ref{sec:hybrid-summary} summarizes the hybrid method and its
% implementation.

\subsection{Multigroup \Gls{SAAF} $S_N$ Neutron Transport Method} \label{sec:saaf}

We implemented the hybrid methods using finite element method numerical solver capabilities available through
Moltres \cite{lindsay_introduction_2018} and the \gls*{MOOSE} framework.
The derivation for the \gls*{SAAF} formulation of the $S_N$ method and its
implementation in Moltres follows closely the derivation developed by Wang et al.\
\cite{wang_diffusion_2014}.

The time-independent, multigroup neutron transport equation defined on the 3-D spatial domain
$\mathcal{D}$ and 2-D unit sphere angular domain $\mathcal{S}$ is:
%
\begin{multline}
  \hat{\Omega}\cdot\nabla\Psi_g(\vec{r},\hat{\Omega},t) + \Sigma_{t,g}
  \Psi_g(\vec{r},\hat{\Omega},t) = \\
  \sum^G_{g'=1}\int_\mathcal{S} \Sigma_s^{g'\rightarrow g}(\hat{\Omega}'\rightarrow\hat{\Omega})
  \Psi_{g'}(\vec{r},\hat{\Omega}',t)d\hat{\Omega}'
  + \frac{1}{4\pi}\frac{\chi_{g}}{k}\sum^G_{g'=1} \nu\Sigma_{f,g'} \phi_{g'}(\vec{r},t)
  \label{eq:mg-nte}
\end{multline}
%
\begin{gather}
  \Psi_g(\vec{r},\hat{\Omega}) = \Psi^\text{inc}_g(\vec{r},\hat{\Omega}) +
  \alpha^s_g\Psi_g(\vec{r},\hat{\Omega}_r)
  \mbox{ on } \vec{r} \in \partial\mathcal{D} \mbox{ and } \hat{\Omega}\cdot\hat{n}_b < 0.
  \label{eq:mg-nte-bc}
%  \shortintertext{where}
%  \begin{align*}
%    \chi_{p,g} &= \mbox{prompt fission neutron spectrum in group $g$,} \\
%    \beta &= \sum^I_{i=1} \beta_i = \mbox{total delayed neutron fraction,} \\
%    \chi_{d,g} &= \mbox{delayed fission neutron spectrum in group $g$,} \\
%    \lambda_i &= \mbox{decay constant of precursor group $i$,} \\
%    C_i &= \mbox{delayed neutron precursor concentration for group $i$,} \\
%    \Psi^\text{inc}_g &= \mbox{incident surface source in group $g$,} \\
%    \alpha^s_g &= \mbox{specular reflectivity on }\partial \mathcal{D} \mbox{ for group }g, \\
%    \hat{\Omega}_r &= \hat{\Omega}-2(\hat{\Omega}\cdot \hat{n}_b)\hat{n}_b, \\
%    \hat{n}_b &= \mbox{outward unit normal vector on the boundary.}
%  \end{align*}
\end{gather}
%
In order to introduce operators and facilitate subsequent mathematical derivations, we will define
the following vector forms:
%
\begin{gather}
  \bm{\Psi} \equiv
  \begin{bmatrix}
    \Psi_1 \\
%    \Psi_2 \\
    \vdots \\
    \Psi_G
  \end{bmatrix},
  \bm{\Phi} \equiv \int_S \bm{\Psi}d\hat{\Omega} \equiv
  \begin{bmatrix}
    \phi_1 \\
%    \phi_2 \\
    \vdots \\
    \phi_G
  \end{bmatrix},
  \bm{\frac{\Psi}{v}} \equiv
  \begin{bmatrix}
    \frac{\Psi_1}{v_1} \\
%    \frac{\Psi_2}{v_2} \\
    \vdots \\
    \frac{\Psi_G}{v_G}
  \end{bmatrix}.
\end{gather}
%
We also define the following operators:
%
\begin{gather}
  \mathbb{L}_1\bm{\Psi} \equiv
  \begin{bmatrix}
    \hat{\Omega}\cdot\nabla\Psi_1 \\
%    \hat{\Omega}\cdot\nabla\Psi_2 \\
    \vdots \\
    \hat{\Omega}\cdot\nabla\Psi_G \\
  \end{bmatrix},
  \mathbb{L}_2\bm{\Psi} \equiv
  \begin{bmatrix}
    \Sigma_{t,1}\Psi_1 \\
%    \Sigma_{t,2}\Psi_2 \\
    \vdots \\
    \Sigma_{t,G}\Psi_G
  \end{bmatrix},
  \mathbb{S}\bm{\Psi} \equiv
  \begin{bmatrix}
    \sum^G_{g'=1}\int_S \Sigma_s^{g'\rightarrow 1}\Psi_{g'}d\hat{\Omega} \\
%    \sum^G_{g'=2}\int_S \Sigma_s^{g'\rightarrow 2}\Psi_{g'}d\hat{\Omega} \\
    \vdots \\
    \sum^G_{g'=G}\int_S \Sigma_s^{g'\rightarrow G}\Psi_{g'}d\hat{\Omega}
  \end{bmatrix}, \nonumber \\
  \mathbb{B}\bm{\Psi} \equiv
  \begin{bmatrix}
    \alpha^s_1\Psi_1(\hat{\Omega}_r) \\
%    \alpha^s_2\Psi_2(\hat{\Omega}_r) \\
    \vdots \\
    \alpha^s_G\Psi_G(\hat{\Omega}_r)
  \end{bmatrix},
  \mathbb{F}\bm{\Psi} \equiv \frac{1}{4\pi}\mathbb{F}_0\bm{\Psi} \equiv
  \begin{bmatrix}
    \frac{1}{4\pi}\chi_{1}\sum^G_{g'=1}\nu\Sigma_{f,g'}\phi_{g'} \\
%    \frac{1}{4\pi}\chi_{2}\sum^G_{g'=1}\nu\Sigma_{f,g'}\phi_{g'} \\
    \vdots \\
    \frac{1}{4\pi}\chi_{G}\sum^G_{g'=1}\nu\Sigma_{f,g'}\phi_{g'}
  \end{bmatrix}.
\end{gather}
%
Eq.\ \ref{eq:mg-nte} \& \ref{eq:mg-nte-bc} can be reexpressed as:
%
\begin{gather}
  \mathbb{L}_1\bm{\Psi} + \mathbb{L}_2\bm{\Psi} = \mathbb{S}\bm{\Psi} + \mathbb{F}\bm{\Psi}, \label{eq:nte-vec} \\
  \bm{\Psi} = \bm{\Psi}^\text{inc} + \mathbb{B}\bm{\Psi}.
\end{gather}

Finally, we present the weak forms of each term in the \gls*{SAAF} formulation of the multigroup
$S_N$ neutron transport equations with void treatment \cite{wang_diffusion_2014} for handling the
$1/\Sigma_{t,g}$ term in near-void regions. We define the inner products
%
\begin{gather}
  \left(\bm{a},\bm{b}\right)_\mathcal{D} \equiv \sum^G_{g=1} \int_\mathcal{D}d\vec{r}
  \sum_{i\in E_e}a_{g,i}b_i(\vec{r})\sum_{j\in E_e}b_{g,j}b_j(\vec{r}), \\
  \left(\bm{a},\bm{b}\right)_{\partial\mathcal{D}} \equiv \sum^G_{g=1}
  \sum_{s\in\partial\mathcal{D}}\int_s d\vec{r}\sum_{i\in E_s}a_{g,i}b_i(\vec{r})\sum_{j\in E_s}
  b_{g,j}b_j(\vec{r}),
\end{gather}
%
involving volume and surface integrals over the spatial domain.
%
\begin{gather}
  \mbox{Time derivative: }
  \left(\left(\mathbb{I}+\bm{\tau}\mathbb{L}_1\right)\bm{\Psi}^*,
  \frac{\partial}{\partial t}\left(\frac{\bm{\Psi}}{\bm{v}}\right)\right) =
  \sum^G_{g=1}\sum^{N_d}_{d=1}w_d\left(\Psi^*_{g,d}+\tau_g\hat{\Omega}_d\cdot\Psi^*_{g,d},
  \frac{\Psi_{g,d}}{v_g}\right)_\mathcal{D} \label{eq:time-derivative} \\
  \mbox{Streaming: }
  \left(\mathbb{L}_1\bm{\Psi}^*,
  \left(\bm{\tau}\mathbb{L}_1-\mathbb{I}+\bm{\tau}\mathbb{L}_2\right)\bm{\Psi}\right) =
  \sum^G_{g=1}\sum^{N_d}_{d=1}w_d\left(\hat{\Omega}_d\cdot\nabla\Psi^*_{g,d},\tau_g\hat{\Omega}
  \cdot\nabla\Psi_{g,d}-(1-\tau_g\Sigma_{t,g})\Psi_{g,d}\right)_\mathcal{D} \\
  \mbox{Collision: }
  \left(\mathbb{L}_2\bm{\Psi}^*,\bm{\Psi}\right) =
  \sum^G_{g=1}\sum^{N_d}_{d=1}w_d\left(\Psi^*_{g,d},\Sigma_{t,g}\Psi_{g,d}\right)_\mathcal{D} \\
  \shortintertext{Scattering:}
  \left(\left(\mathbb{I}+\bm{\tau}\mathbb{L}_1\right)\bm{\Psi}^*,\mathbb{S}\bm{\Psi}\right) =
  \sum^G_{g=1}\sum^{N_d}_{d=1}w_d\left(\Psi^*_{g,d}+\tau_g\hat{\Omega}_d\cdot\nabla\Psi^*_{g,d},
  \sum^G_{g'=1}\sum^L_{l=0}\Sigma^{g'\rightarrow g}_{s,l}\sum^l_{m=-l}
  \frac{2l+1}{w}Y_{l,m}(\hat{\Omega}_d)\phi_{g',l,m}\right)_\mathcal{D} \\
  \mbox{Fission source: }
  \left(\left(\mathbb{I}+\bm{\tau}\mathbb{L}_1\right)\bm{\Psi}^*,\mathbb{F}\bm{\Psi}\right) =
  \sum^G_{g=1}\sum^{N_d}_{d=1}w_d\left(\Psi^*_{g,d}+\tau_g\hat{\Omega}_d\cdot\nabla\Psi^*_{g,d},
  \frac{1}{w}\frac{\chi_g}{k}\sum^G_{g'=1}\nu\Sigma_{f,g'}\phi_{g'}\right)_\mathcal{D}
\end{gather}
%
$\tau_g$ is a numerical stabilization parameter borne from the void treatment scheme.
The vacuum, boundary source, and reflecting boundary terms are given as:
%
\begin{gather}
  \langle\bm{\Psi}^*,\bm{\Psi}\rangle^+ - \langle\bm{\Psi}^*,\bm{\Psi}^\text{inc}\rangle^- =
  \begin{cases}
    \sum^G_{g=1}\sum^{N_d}_{d=1}w_d\left(\Psi^*_{g,d},
    \hat{\Omega}_d\cdot\hat{n}_b\Psi_{g,d}\right)_{\partial\mathcal{D}},
    & \hat{\Omega}\cdot\hat{n}_b>0,\vec{r}\in\partial\mathcal{D} \\
    0,
    & \hat{\Omega}\cdot\hat{n}_b<0,\vec{r}\in\partial\mathcal{D}
  \end{cases}, \\
  \langle\bm{\Psi}^*,\bm{\Psi}\rangle^+ - \langle\bm{\Psi}^*,\bm{\Psi}^\text{inc}\rangle^- =
  \begin{cases}
    \sum^G_{g=1}\sum^{N_d}_{d=1}w_d\left(\Psi^*_{g,d},
    \hat{\Omega}_d\cdot\hat{n}_b\Psi_{g,d}\right)_{\partial\mathcal{D}},
    & \hat{\Omega}\cdot\hat{n}_b>0,\vec{r}\in\partial\mathcal{D} \\
    \sum^G_{g=1}\sum^{N_d}_{d=1}w_d\left(\Psi^*_{g,d},
    \hat{\Omega}_d\cdot\hat{n}_b\Psi^\text{inc}_{g,d}\right)_{\partial\mathcal{D}},
    & \hat{\Omega}\cdot\hat{n}_b<0,\vec{r}\in\partial\mathcal{D}
  \end{cases}, \label{eq:boundary-source} \\
  \langle\bm{\Psi}^*,\bm{\Psi}\rangle^+ - \langle\bm{\Psi}^*,\mathbb{B}\bm{\Psi}\rangle^- =
  \begin{cases}
    \sum^G_{g=1}\sum^{N_d}_{d=1}w_d\left(\Psi^*_{g,d},
    \hat{\Omega}_d\cdot\hat{n}_b\Psi_{g,d}\right),
    & \hat{\Omega}\cdot\hat{n}_b>0,\vec{r}\in\partial\mathcal{D}_s \\
    \sum^G_{g=1}\sum^{N_d}_{d=1}w_d\left(\Psi^*_{g,d},
    \hat{\Omega}_d\cdot\hat{n}_b\Psi_{g,d_r}\right),
    & \hat{\Omega}\cdot\hat{n}_b<0,\vec{r}\in\partial\mathcal{D}_s
  \end{cases}, \label{eq:reflecting-bc}
\end{gather}
%
where $\hat{\Omega}_{d_r} = \hat{\Omega}_d - 2(\hat{\Omega}_d\cdot\hat{n}_b)\hat{n}_b$.

\subsection{Transport-Corrected Neutron Diffusion Method} \label{sec:transport-correction}

% In this work, we explored two options for applying transport corrections to the neutron
% diffusion equations: a diffusion correction scheme and a drift correction scheme.
% We performed investigations on 1-D reactor models with both transport correction
% schemes but eventually elected to use the drift correction scheme for 2-D reactor models
% due to superior numerical properties.
% We implemented the drift correction-based hybrid
% method on Moltres \cite{lindsay_moltres_2017} and the diffusion correction-based hybrid method
% using Python.

% \subsubsection{Diffusion Correction Term} \label{sec:diffusion-correction}
% 
% Diffusion corrections involve replacing the diffusion coefficient $D_g$ in the
% diffusion term with ``optimal'' diffusion coefficients based on
% pointwise transport corrections. We used a formulation that incorporates pointwise
% corrections to the neutron diffusion flux solution from the $S_N$-derived flux solution as follows:
% %
% \begin{align}
%   D^s_g(x) &= -J^{tr}_g(x)\bigg/\frac{d\phi^{tr}_g(x)}{dx}. \label{eq:svdc}
% \end{align}
% %
% where $D^s_g$ is the diffusion correction parameter, and the $tr$ superscript denotes the
% transport-derived neutron
% current and scalar flux solutions from the $S_N$ method. Transport corrections introduced through
% $J_{tr}$ are scaled by the flux gradient. We assumed that it varies continuously and is
% at least once differentiable except at dissimilar material interfaces. $D^s_g$ provides
% pointwise corrections to closely match the diffusion flux solution to the $S_N$ flux solution.
% By replacing $D_g$ with $D^s_g$, we are effectively adding the following transport correction term
% %
% \begin{gather}
%   -\frac{\partial}{\partial x}(D^s_g-D_g)\frac{\partial\phi_g}{\partial x}
% \end{gather}
% to the neutron diffusion equations. Alternatively, we may define
% $\partial D\equiv\left(D^s_g-D_g)\right)$. This alternative form shows that diffusion
% correction applies a multiplicative closure that scales with the flux gradient.

% \subsubsection{Drift Correction Term} \label{sec:drift-correction}

The general form of drift correction terms to be added to the neutron diffusion equations is a
first-order derivative term $\nabla\cdot \vec{D}_g\phi_g$.
Derivations for the drift terms depend on the discretization schemes of the high- and low-order
equations. For this work, we adopted drift terms derived by Wang et al.\
\cite{wang_diffusion_2014}. The neutron balance equation derived from
integrating \gls*{SAAF} equation is:
%
\begin{multline}
  \left(\bm{\Phi}^*,\frac{\partial}{\partial t}\left(\frac{\bm{\Phi}}{\bm{v}}\right)\right)_\mathcal{D}
  + \left(\nabla\bm{\Phi}^*, \mathbb{D}\nabla\bm{\Phi}\right)_\mathcal{D}
  + \left(\mathbb{L}_2\bm{\Phi}^*,\bm{\Phi}\right)_\mathcal{D}
  + \left(\bm{\Phi}^*,\frac{\bm{\Phi}}{2}\right)_{\mathcal{D}_v}
  + \left(\nabla\bm{\Phi}^*,\bm{\tau}\vec{\bm{r}}_1-\vec{\bm{J}}-\mathcal{D}\bm{\Phi}\right)_\mathcal{D}
  + \\
  \left(\bm{\Phi}^*,\bm{J}^\text{out}-\frac{\bm{\Phi}}{2}\right)_{\partial\mathcal{D}_v}
  = \left(\bm{\Phi}^*,\mathbb{S}_0\bm{\Phi}\right)_\mathcal{D}
  + \left(\bm{\Phi}^*,\bm{Q}_0\right)_\mathcal{D}. \label{eq:modified-diff}
\end{multline}
%
We can interpret Eq.\ \ref{eq:modified-diff} as a modified neutron diffusion equation with
transport corrections provided by a drift term with the drift vector defined as
%
\begin{gather}
  \vec{\bm{D}} \equiv \frac{\bm{\tau}\vec{\bm{r}}_1-\vec{\bm{J}}-\mathbb{D}\nabla\bm{\Phi}}{\bm{\Phi}}
\end{gather}
%
and a vacuum boundary correction term with the boundary coefficient vector defined as
%
\begin{gather}
  \bm{\gamma} \equiv \frac{\bm{J}^\text{out}}{\bm{\Phi}}-\frac{1}{2}\bm{I}.
\end{gather}
%
With the $S_N$ angular discretization scheme, the drift and boundary correction vector components
can be evaluated as
%
\begin{gather}
  \vec{D}_g = \frac{\sum^{N_d}_{d=1}w_d\left(\tau_g\hat{\Omega}_d\hat{\Omega}_d\cdot\nabla\Psi_{g,d}
  + \left(\tau_g\Sigma_{t,g}-1\right)\hat{\Omega}_d\Psi_{g,d}
  - \tau_g\sum^G_{g'=1}\Sigma^{g'\rightarrow g}_{s,1}\hat{\Omega}_d\Psi_{g',d}
  - D_g\nabla\Psi_{g,d}\right)}{\sum^{N_d}_{d=1}w_d\Psi_{g,d}}, \label{eq:drift} \\
  \gamma_g =
  \frac{\sum_{\hat{\Omega}_d\cdot\hat{n}_b > 0}w_d |\hat{\Omega}_d\cdot\hat{n}_b |
  \Psi_{g,d}}{\sum^{N_d}_{d=1}w_d\Psi_{g,d}}. \label{eq:bound-coef}
\end{gather}

\subsection{$S_N$-Diffusion Iteration Algorithm \& Boundary Coupling} \label{sec:hybrid-algorithm}

In order to reduce the computational cost of the high-level $S_N$ calculation in a reactor
simulation, we propose reducing the problem domain of the $S_N$ method to a
\textit{correction subregion} containing the control rod
and its vicinity. Consequently, the hybrid $S_N$-diffusion method may retain accurate neutron flux
and current estimates around the control rod region from the $S_N$ method while making significant
computational cost savings by treating most of the reactor geometry with the neutron diffusion
method alone. Henceforth, we will refer to the $S_N$ calculation on the correction
region as the $S_N$ \textit{subproblem} or \textit{subsolver}. We define the full problem
domain and the correction subregion as $V_0$ and $V_1$, respectively, where
$V_1\subseteq V_0$. The iterative algorithm for the hybrid $S_N$-diffusion method is as follows:
%
\begin{enumerate}
  \item Start with an initial neutron diffusion calculation in $V_0$ using the standard neutron
    diffusion method.
  \item Pass the neutron diffusion current estimates along
    $\partial V_1$ to the $S_N$ subsolver to evaluate the boundary conditions for the $S_N$
    subproblem.
  \item With the $S_N$ subsolver, evaluate transport correction terms in $V_1$ using Eq.\
    \ref{eq:drift}.
  \item Pass the transport correction terms to the neutron diffusion solver to apply corrections
    within $V_1$ while continuing to apply the standard neutron diffusion solver
    in the rest of $V_0$.
  \item Start the next iteration by running a neutron diffusion calculation with transport
    corrections in $V_1$.
  \item Repeat Steps 2-6 until convergence is reached by meeting desired convergence tolerance
    values.
\end{enumerate}

For the hybrid $S_N$-diffusion method to converge, it requires appropriate boundary conditions for
the $S_N$ subproblem.
Given that we want to limit the coverage of $V_1$ to the control rod region and its
immediate vicinity, $V_1$ should be sufficiently smaller than $V_0$, but large enough to capture
anisotropies in the flux due to the control rod. As a consequence, the boundaries $\partial V_1$ 
lie well within $V_0$. We may obtain the best boundary source estimate for the $S_N$ subproblem by
applying the $P_1$ approximation for evaluating the neutron angular flux along
the discrete ordinates $\hat{\Omega}_d$ of the $S_N$ angular quadrature set as follows
%
\begin{align}
  \Psi_{g,d} \approx \frac{1}{4\pi}\left(\phi^\text{diff}_g+3\hat{\Omega}_d\cdot
  \vec{J}^\text{diff}_g\right)
  =\frac{1}{4\pi}\left(\phi^\text{diff}_g-3\hat{\Omega}_d\cdot D_g\nabla\phi^\text{diff}_g\right)
\end{align}
%
where the diff superscript denotes flux estimates from the latest neutron diffusion calculation.
With this relation, we can formulate the boundary source term for the $S_N$ subsolver by setting
$\Psi^\text{inc}_{g,d}$ in Eq.\ \ref{eq:boundary-source} to
%
\begin{gather}
  \Psi^\text{inc}_{g,d} = \frac{1}{w}
  \left(\phi^\text{diff}_g-3\hat{\Omega}_d\cdot D_g\nabla\phi^\text{diff}_g\right)
\end{gather}

The angular flux solution to the $S_N$ subproblem defined with these boundary sources typically
produces accurate drift correction parameters within $V_1$ but tend to be inaccurate
near $\partial V_1$. Consequently, we define a \textit{buffer region}
$V_1'$, within $V_1$ adjacent $\partial V_1$, where inaccurate drift correction parameters are
discarded. Right before each
neutron diffusion iteration, the solver adaptively determines a cutoff boundary for $V_1'$ which
determines the extent in $V_1$ where the drift correction terms are applied.
A natural/intuitive criterion for the location of the cutoff boundary
would be wherever the components of the drift correction variable $\vec{D}_g$ is zero, i.e.,
wherever the components change signs. The reasons for this criterion are twofold. Firstly, at
points where the $\vec{D}_g$ components are zero, the flux is approximately isotropic along the
axes corresponding to the components because value of $\vec{D}_g$ represents how much correction a
standard neutron diffusion model requires. Secondly, this choice preserves the smoothness of the
neutron scalar flux gradient.

\section{Model Setup}

We modeled the 1-D models in this work after the \gls*{MSRE} design \cite{robertson_msre_1965}. We
employed the OpenMC Monte
Carlo neutral particle transport software \cite{romano_openmc:_2015} for generating group constants
for the hybrid method and reference solutions for verification. All multigroup
neutronics simulations ran with an eight neutron energy group structure developed by Jaradat
\cite{jaradat_development_2021-1}. The OpenMC models ran on continuous energy (OpenMC-CE) and
multigroup (OpenMC-MG) modes and utilized the ENDF/B-VII.1 nuclear data library.

This work evaluates six 1-D test cases with increasing complexity to verify the $S_N$ and hybrid
$S_N$-diffusion implementations and to test the performance of the hybrid
method in response to various geometrical features. The last two cases resemble the
reference \gls*{MSRE} design, which has centrally located control rods
and air-filled rod guide tubes. Figure \ref{fig:case-geom} shows the geometries of Cases 1a to
3b. All geometries have reflective boundary conditions at $x=0$ cm, thereby
forming half-core or repeating unit cell models. Cases 1a and 1b are repeating unit
cell models with reflecting boundaries on the right-side boundaries. Cases 2a, 2b, 3a, and 3b
are 1-D half-core models with vacuum boundaries on the right-side boundaries.

All material compositions and densities, except the control rod, follow
reference data at 900 K from \gls*{MSRE} benchmark specifications \cite{fratoni_molten_2020}.
We reduced the Gd$_2$O$_3$ content in the control rod region from 70 wt\% to 0.35 wt\% to bring the
rod worth down from approximately 55000 pcm to 20000 pcm in the 1-D models.
%
\begin{figure}[htb!]
  \centering
  \includegraphics[width=0.75\columnwidth]{case-geometry}
  \caption{Geometries of the 1-D test cases. The material labeled ``mixture'' represents a
    homogeneous mixture of fuel and graphite at a ratio of 22.5\%-77.5\% by volume. All geometries
    have reflective boundary conditions on the boundary at $x=0$ cm. The right-side boundaries are
    reflective for Cases 1a and 1b, and vacuum for Cases 2a, 2b, 3a, and 3b.}
  \label{fig:case-geom}
\end{figure}

\section{Numerical Results \& Discussion}

The hybrid method shares some similarities with diffusion-based acceleration schemes
for the neutron transport methods which do not require the neutron transport side of the
calculations to fully converge \cite{wang_diffusion_2014}. The typical convergence threshold value
required for converged 1-D neutron flux calculations is $10^{-8}$. We varied the convergence
tolerance value for the $S_N$ subsolver to find the optimal value for maintaining sufficient
solution accuracy and minimizing computational costs.
Table \ref{table:sn-tol} lists the number of outer iterations required by the hybrid method for
a given set of $S_8$ subsolver convergence tolerance values in Case 3b.

\begin{table}[htb]
  \begin{minipage}{0.45\columnwidth}
    \centering
    \small
    \caption{Number of outer iterations in hybrid method calculations of Case 3b for a given set of
    convergence tolerance values imposed on the $S_8$ subsolver.}
    \begin{tabular}{S S} % S S S S S}
  %    \toprule
  %    $S_8$ subsolver tolerance, $\epsilon_\text{tol}$ & {$10^{-8}$} & {$10^{-7}$} & {$10^{-6}$} & {$10^{-5}$} & {$10^{-4}$} & {$10^{-3}$} \\
  %    \midrule
  %    Number of outer iterations & 3 & 3 & 3 & 2 & 2 & 1 \\
  %    \bottomrule
      \toprule
      {$S_8$ subsolver tolerance} & {No.\ of outer iterations} \\
      \midrule
      {$10^{-8}$} & 3 \\
      {$10^{-7}$} & 3 \\
      {$10^{-6}$} & 3 \\
      {$10^{-5}$} & 2 \\
      {$10^{-4}$} & 2 \\
      {$10^{-3}$} & 1 \\
      \bottomrule
    \end{tabular}
    \label{table:sn-tol}
  \end{minipage}
  \begin{minipage}{0.55\columnwidth}
    \centering
    \includegraphics[width=0.75\columnwidth]{sn-tol}
    \captionof{figure}{$k_\text{eff}$ error estimates of Case 3b for a range of $S_8$ subsolver
    convergence tolerance values. The $q=1.333$ line represents the approximate rate of
    convergence.}
    \label{fig:sn-tol}
  \end{minipage}
\end{table}

Figure \ref{fig:sn-tol} shows the $k_\text{eff}$ error estimates relative to the reference value
when $\epsilon_\text{tol}=10^{-8}$. The hybrid method exhibits superlinear convergence ($q=1.333$)
with respect to the $S_N$ subsolver convergence tolerance value. Based on the results, setting
$\epsilon_\text{tol}$ to $10^{-5}$ is the best balance between accuracy and computational
cost because the $k_\text{eff}$ error is less than 0.01 pcm and the number of outer iterations
would increase from 2 to 3 if the tolerance value is further tightened.

% \begin{figure}[htb!]
%   \centering
%   \includegraphics[width=0.4\columnwidth]{sn-tol}
%   \caption{$k_\text{eff}$ error estimates of Case 3b for a range of convergence tolerance values
%   imposed on the $S_8$ subsolver relative to the reference $k_\text{eff}$ value when
%   $\epsilon_\text{tol}=10^{-8}$. The $q=1.333$ line represents the approximate rate of
%   convergence.}
%   \label{fig:sn-tol}
% \end{figure}

Figure \ref{fig:1d-rho} shows the difference in reactivities of OpenMC-MG, $S_8$, neutron
diffusion, and hybrid methods relative to OpenMC-CE for all test cases. The standard deviations of
reactivity values from OpenMC-CE are approximately 40 pcm for Cases 1a and 1b and 60 pcm for the
rest as depicted by the blue highlighted area in Figure \ref{fig:1d-rho}. For Cases 1a and 1b, the
OpenMC-MG, $S_8$, and neutron diffusion methods show excellent agreement with OpenMC-CE as the
reactivity values fall within the one or two standard deviation range.
%
\begin{figure}[htb!]
  \centering
  \includegraphics[width=0.75\columnwidth]{rho}
  \caption{Difference in reactivity $\rho$ of all neutronics methods investigated relative
  to OpenMC-CE.}
  \label{fig:1d-rho}
\end{figure}

For all cases, the OpenMC-MG and the $S_8$ methods show consistent agreement with one another. 
While they deviate from OpenMC-CE by approximately 350 pcm for Cases 2a and 3a, they fall within
two standard deviations of OpenMC-CE for Cases 2b and 3b. When increasing the maximum Legendre
polynomial order to approximate the angular dependence in $\Sigma_s^{g'\rightarrow g}$ from
2nd-order to 3rd-order, the reactivity changed by only 1 pcm. Therefore, we attribute the reactivity
discrepancies to the eight neutron energy group structure which remains the
only significant difference between OpenMC-CE and the multigroup methods.

The neutron diffusion and hybrid methods agree closely with one another for Cases 2a and 3a which
exclude the control rod region. This shows that the hybrid method provides similar $k_\text{eff}$
estimates to the neutron diffusion method in 1-D simulations without highly neutron-absorbing
regions. Compared to the OpenMC-MG and $S_8$ reactivity values, the neutron diffusion and hybrid
method reactivity values agree closer with the reference OpenMC-CE value. However, this is likely
due to favorable error cancellation since diffusion theory is an approximation to neutron
transport.

In Cases 2b and 3b, the neutron diffusion method largely fails to accurately capture the
effect of introducing the control rod region as evidenced by the -1500 pcm and -1150 pcm
discrepancies relative to OpenMC-CE. The hybrid method fares better with -400 pcm and -300 pcm
discrepancies.

Moving the discussion to control rod worths, figure \ref{fig:1d-worth} shows the percentage
difference in rod worths for Cases 2 and 3 of OpenMC-MG, $S_8$, neutron diffusion, and hybrid
methods relative to OpenMC-CE. We calculated the rod worth estimates from each method by taking the
difference in reactivity between the ``rod out'' (a) and ``rod in'' (b) configurations of Cases 2
and 3. All neutronics methods overestimate the rod worth relative to OpenMC-CE, resulting in
positive percentage difference values observed in the figure.
The neutron diffusion methods clearly show up as outliers with rod worth estimates in excess
of 8\%. The remaining methods cluster around 2.5\% and 3\% percentage difference for Cases
2 and 3.

Given that the hybrid methods rely on transport corrections derived from the $S_N$ method, the
$S_8$ rod worth estimates serve as the
reference point for hybrid method verification. Additionally, errors arising from the multigroup
approximation affect both hybrid and $S_N$ simulations to similar degrees. The hybrid method
provides significant improvements in rod worth estimates over the neutron diffusion method.
%
\begin{figure}[htb!]
  \centering
  \includegraphics[width=0.75\columnwidth]{worth}
  \caption{Percentage difference in rod worth for Cases 2 and 3 of all neutronics methods
  investigated relative to OpenMC-CE.}
  \label{fig:1d-worth}
\end{figure}

Figure \ref{fig:3b-flux-diff} shows the absolute
difference in Case 3b neutron group flux distributions of the $S_8$, neutron diffusion, and
hybrid methods relative to OpenMC-MG, instead of OpenMC-CE, to eliminate discrepancies arising from
the multigroup approximation. Case 3b most closely resembles the actual \gls*{MSRE} design with an
inserted rod.
The neutron diffusion and hybrid methods fare worse than the $S_8$ method at capturing the
oscillatory pattern in groups 1, 2, 5, 7, and 8 arising from the fuel-graphite lattice geometry
along $x=5$ cm to 10 cm. The neutron diffusion method
exhibits significant flux deviations in groups 1, 2, 7, and 8 near $x=0$ cm. The hybrid method
provides significant improvements in neutron flux distributions compared to the neutron diffusion
method as shown by the generally smaller flux difference values particularly near $x=0$ cm.

\begin{figure}[htb!]
  \centering
  \includegraphics[width=\columnwidth]{case-3b-flux-diff}
  \caption{Absolute difference in neutron group flux distributions for Case 3b from $S_8$,
  neutron diffusion, and hybrid methods relative to OpenMC-MG.}
  \label{fig:3b-flux-diff}
\end{figure}

Figure \ref{fig:3b-drift} shows the drift correction parameter distribution in each neutron energy
group for Case 3b. We computed the reference drift distributions from reference flux
solutions of the standard $S_8$ method.
The drift correction parameters generated by the correction scheme match their respective reference
values within the correction subregion from $x=0$ cm to 10 cm except near the subregion boundary at
$x=10$ cm as previously discussed in Section \ref{sec:hybrid-algorithm}. The adaptively defined
buffer region $V_1'$ completely encompasses the region near $x=10$ cm where the drift correction
distributions deviate from the reference distribution and are thus discarded.

\begin{figure}[htb!]
  \centering
  \includegraphics[width=\columnwidth]{case-3b-drift}
  \caption{Multigroup drift correction ($\vec{D}_g$) $x$-component distributions from the
  Moltres-hybrid and Moltres-$S_8$ solvers.}
  \label{fig:3b-drift}
\end{figure}

% The hybrid $S_N$-diffusion method relies on $S_N$ calculations in the correction subregion to
% generate flux corrections. Minimizing the size of the correction subregion and the $S_N$ subproblem
% is essential for the hybrid method to be computationally competitive for time-dependent full-core
% simulations. I investigated the effect of the correction subregion size on the $k_\text{eff}$ and
% control rod worth estimates with Cases 3a and 3b by varying the subregion sizes from 10 cm to 40 cm
% at 5 cm-intervals.
% 
% Figures \ref{fig:v1-size-a-k} and \ref{fig:v1-size-b-k} show the $k_\text{eff}$ estimates from the
% hybrid method for Cases 3a and 3b, respectively. In both cases, the $k_\text{eff}$ values initially
% decrease as the correction subregion sizes increase before reversing in trend when the subregion
% size reaches 35 cm and beyond. The $k_\text{eff}$ values vary by up to 164 pcm for Case 3a and 109
% pcm for Case 3b. An important observation here is that the $k_\text{eff}$ values do not
% monotonically converge towards the $k_\text{eff}$ estimate from the $S_8$ method. The data implies
% other significant sources of discrepancies exist beyond those being corrected in the
% correction subregion by the hybrid method.
% 
% \begin{figure}[htb!]
%   \centering
%   \includegraphics[width=\columnwidth]{correction-size-b-k}
%   \caption{$k_\text{eff}$ estimates from the hybrid method for Case 3b with different
%   correction subregion sizes. The horizontal lines indicate $k_\text{eff}$ estimates from the
%   OpenMC-CE, OpenMC-MG, and $S_8$ methods.}
%   \label{fig:v1-size-k}
% \end{figure}
% 
% Figure \ref{fig:v1-size-rho} shows the percentage difference in rod worth relative to OpenMC-CE. 
% Due to the identical trends observed in the $k_\text{eff}$ estimates of both cases, the control rod
% worth estimates do not change significantly when the correction subregion
% sizes change. This indicates limiting the correction subregion size to save
% computational cost on the expensive $S_N$ calculations has a negligible impact on rod worth
% estimates. The rod worth estimates for all investigated correction subregion sizes remain within
% 0.2\% of the $S_8$ method.
% 
% \begin{figure}[htb!]
%   \centering
%   \includegraphics[width=0.8\columnwidth]{correction-size-rho}
%   \caption{Percentage difference in rod worth from the hybrid method relative to OpenMC-CE for
%     Cases 3a and 3b with different correction subregion sizes. The horizontal lines indicate
%     equivalent rod worth differences from the OpenMC-MG and $S_8$ methods.}
%   \label{fig:v1-size-rho}
% \end{figure}

\section{CONCLUSIONS}

This paper presents a hybrid $S_N$-diffusion method aimed at control rod modeling in time-dependent
simulations. The method involves generating drift correction parameters within a control rod and
its vicinity using the $S_N$ method for the neutron diffusion method. The transport-corrected
subregion size adaptively changes in response to the drift correction distributions. We implemented
this method alongside the preexisting neutron diffusion solver in Moltres, a \gls*{MOOSE}-based
application. The reactivity and rod worth results in \gls*{MSR}-based 1-D test cases show the hybrid
method accurately reproduces rod worth estimates. The drift correction parameter
distributions indicate that the influence of the rod on the drift, and
consequently the flux, extend to approximately $x=5$ cm before the drift distribution settles on a
regular repeating pattern influenced by the fuel-graphite lattice structure. Further work is
ongoing for the extension of the hybrid method to 2-D and 3-D models and application to
time-dependent simulations.

\section*{ACKNOWLEDGEMENTS}
Sun Myung Park was supported by a research assistantship provided by the Nuclear, Plasma, \&
Radiological Engineering (NPRE) department at the University of Illinois Urbana-Champaign (UIUC).

\bibliographystyle{mc2025}
\bibliography{bibliography}

\end{document}
